\documentclass[11pt]{article}
\usepackage[margin=1in]{geometry}
\usepackage{xcolor}

\begin{document}

\newpage
\section*{Editors’ comments}
I have now received two reviews of your paper. Both reviewers were excited about the methodological concept, but both where also concerned about 1) the perceived narrow application of the methodology and 2) the example based on data which are not accessible. The reviewers have suggested ways to perhaps broaden the potential applications of the method and have even suggested additional (alternative) data sets that ARE accessible.  I strongly encourage you to utilize data that others can access to explore your methods. If you want to make insights on HIV evolution using restricted data, then please send such work to a medical journal. But for MBE, the expectation (especially for a methodological paper) is that the demonstrations of methods should be using data that are easily and openly accessible.  Hopefully, you will find these reviews helpful in revising your work.

\textcolor{blue}{We thank the editor for the chance to improve our manuscript and the two reviewers for their constructive feedback. We feel the quality of the manuscript improved thanks to their suggestions. Below, we highlight the main improvements in the revised manuscript. We also discuss two suggestions we did not implement. Finally, in the point-by-point responses, we highlight the specific changes to the manuscript based on each of the reviewers comments.}

\textcolor{blue}{The main improvement in the revised manuscript is that we added an analysis of an open dataset and re-phrased our abstract and introduction to focus on the broader applicability of our method, including what differentiates it from previous methods. We highlight that our two-step method provides additional insights on the evolutionary dynamics of a trait in a pathogen population and that it is compatible with a variety of downstream GWAS association testing methods. We removed the HIV-specific information from the introduction and moved the relevant comparisons to the discussion section.}

\textcolor{blue}{The only suggestions we did not implement were to re-run the HIV GWAS using a linear mixed model and to perform the HIV analysis again using elite controllers. In the first case, we rephrased the discussion to make it clearer that our method is envisioned as a pre-processing step that can be used prior to GWAS association testing under any model. Given the revised focus of the manuscript away from HIV-1 specific insights, we felt improving the HIV-1 GWAS association testing methodology was beyond the scope of the current manuscript. In the second case, our dataset does not contain enough elite controllers to perform the additional analysis.}

\section*{Reviewer 1}

The authors present a new GWAS method that estimates and removes trait variability due to the pathogen using information from the full pathogen phylogeny. I do not have critiques about the framework, as the method the authors have developed seems novel and in principal, groundbreaking. However, I am finding difficult to grasp the usability of it outside of pathogens that integrates in host genome. What is the broad applicability of the method? Is this method only applicable to stVL in the context of HIV? What else could be investigated? Is it possible to pair transcriptomic data? Can author expand on this?

\vspace{5mm}
\textcolor{blue}{We edited the introduction to make it clearer when this method is applicable: ``The introduced framework relies on paired pathogen-host genotyping and is envisioned specifically for continuous-valued traits that are highly heritable from infection partner to infection partner.''}

\textcolor{blue}{Then, as mentioned, we included a second application of our method to a phytopathogenic bacteria.}

\textcolor{blue}{We also expanded the discussion section that gives more specific examples of future potential applications, including the possibility of transcriptomic applications, which we hadn't previously thought about but should be possible: ``Future applications of our method might investigate other clinically significant disease traits and outcomes that are affected by both host and pathogen genetic factors, for instance Hepatitis B Virus-related hepatocellular carcinoma (An et al., 2018), Hepatitis C treatment success (Ansari et al., 2017), and susceptibility to or severity of certain bacterial infections, e.g. Donnenberg et al. (2015); Messina et al. (2016). Transcriptomic data has also previously been modeled as an evolving phenotype using an Ornstein-Uhlenbeck model (Rohlfs et al., 2014). Thus, one could also estimate pathogen effects on host gene expression.''}

\begin{enumerate}
    \item It would have been interesting to see how their methods performs with another type of pathogen, a virus that does not integrate, such SARS-CoV-2, or a bacterial one, such TB. \\
    \textcolor{blue}{We added an analysis of \emph{A. thaliana} quantitative disease resistance to the phytopathogenic bacteria \emph{X. arboricola}. The findings are highlighted in the discussion: ``Given our estimates for trait heritability and selection strength on HIV-1 spVL and A. thaliana quantitative disease resistance to X. arboricola, our simulation results reveal that we cannot expect a significant improvement in GWAS power for these systems (Figure 2)... For A. thaliana QDR to X. arboricola, the trait value correction does not utilize phylogenetic information because phylogenetic correlations between samples are too weak (maximum expected correlation between strains was $3.2 \times 10^{-12}$). We anyways corrected QDR trait values based on average QDR for each pathogen strain across the full range of host types. Results show slight decrease in p-values for the most-associated variants in this application as well, but the overall picture is consitent with previous GWAS results from Wang et al. (2018).''}
    \item In the case of co-infection, as sometimes is the case of HIV and TB, how would the method perform? \\
    \textcolor{blue}{We addressed this point in the discussion: ``Further, additional model complexity can be added to the GWAS association tests. For instance, our method does not account for co-infection, which might add additional variance to trait values and decrease GWAS power. In this case, one could add co-infection status as a covariate in the GWAS association test to account for this variable.''}
    \item In the case of HIV, is it possible to compare the data with elite controllers? \\
    \textcolor{blue}{Unfortunately our dataset does not include a sufficient number of elite controllers to perform this comparison. The SHCS cohort includes 83 individuals classified as elite controllers, as defined by at least three measurements of plasma virus load (VL) $<$ 2000 RNA copies/ml over at least a 12-month period in the absence of antiviral therapy (Pereya et al, Science, 2010). spVL values range from 1.3 to 3.1 in this group (all very low). After filtering steps, only 8 of these individuals were retained for the GWAS, so the sample size is unfortunately not sufficient to do a comparative GWAS. However, we note that McLaren et al. (2015) found similar signal of association on chromosome 3 and 6 when they filtered to elite controllers in a standard GWAS setup.}
    \item In the case of determining whether viral effects may be confounding or extraneous variables that bias estimates of host genetic effects, I am not sure I understand the premise of this hypothesis. Can authors elaborate on it more in detail? \\
    \textcolor{blue}{We edited the introduction to remove the confounding and extraneous variables terminology and make it clearer why one should consider pathogen effects in a host GWAS: ``Shared ancestry among individuals, especially between close relatives, can give rise to spurious genetic correlations with a trait. Corrections for these types of population structure in human GWAS cohorts are well- developed and widely accepted (Astle and Balding, 2009; Price et al., 2006). More recently, analogous methods have been developed for microbial GWAS, where clonal reproduction exacerbates population structure (Power et al., 2017).'' We also make it clearer how our method addresses this problem: ``In a first step, we fit an evolutionary model to trait data and the pathogen phylogeny. This first step provides an estimate of the correlation structure of the trait due to heritable pathogen effects. The estimate is used to remove pathogen effects on the trait. In the second step, the resulting corrected trait data is used in a GWAS with host genetic variants. The GWAS can be performed as normal under the assumption of independent samples.''}
    \item What type of clinical information/trait is included under the environmental effect? How the diversity in environmental effects across the population affects the method? \\
    \textcolor{blue}{We edited the text to give examples of factors that would be grouped into the environmental effect under the POUMM model we use: ``Variation in infectious disease traits like viral load or infection severity can come from several sources. These include host genetic factors, pathogen genetic factors, interaction effects between the host and the pathogen, or non-genetic factors like healthcare quality or temperature... In the infectious disease GWAS case, we assume the heritable part comprises pathogen genetic factors and all other factors are non-heritable.''}
    
    \textcolor{blue}{We also added text to the new approaches section to make it clearer that our method is complementary to but does not replace other methods to account for diversity in environmental effects (we only account for heritable pathogen genetic effects): ``GWAS typically stratify samples or include covariates to correct for host genetic factors or non-genetic factors that may be correlated with a trait value. This leaves pathogen genetic factors as a remaining source of correlation, since close transmission partners may be infected with very similar pathogen strains. We aim to remove this pathogen-induced correlation in the trait data prior to performing GWAS on the host genomes.''}
    \item Lines 194-196:  please expand on the parameters of the simulation in greater detail. \\
    \textcolor{blue}{We added a full description of the simulation scheme: ``We first simulated a phylogeny of 500 tips with exponentially distributed branch lengths and mean root-to-tip time of 0.14 substitutions per site per year as in Hodcroft et al. (2014)... Finally, we sampled an additional random environmental effect for each tip from a normal distribution centered at 0, as illustrated in Figure 1C.''}  
    \item Lines 205-207: it would be best, if there is not a limitation in the number of words, to report here “why” instead of saying that it has been shown in SI. \\
    \textcolor{blue}{We added the suggested explanation: ``As shown in the supplemental material, we can calculate the expected RMSE using the scaled trait value across scenarios in our simulation scheme because the variance in the trait due to pathogen genetic effects and environmental effects is fixed. Thus, we expect the RMSE using the scaled trait value to be 0.74 across all simulation scenarios.''}
    \item Liens 212: “our method performs well when an infectious disease trait is highly heritable”, I found the concept of heritability in the infectious disease filed confusing. Are the authors talking about traits in the viral population? Is it possible to expand the sentence with more details for clarity? \\
    \textcolor{blue}{We edited this statement to be clear that we mean ``pathogen heritability''. We also tried to explain the concept of heritability due to the pathogen better in the New Approaches section: ``Variation in infectious disease traits like viral load or infection severity can come from several sources. These include host genetic factors, pathogen genetic factors, interaction effects between the host and the pathogen, or non-genetic factors like healthcare quality or temperature. GWAS typically stratify samples or include covariates to correct for host genetic factors or non-genetic factors that may be correlated with a trait value. This leaves pathogen genetic factors as a remaining source of correlation, since close transmission partners may be infected with very similar pathogen strains. We aim to remove this pathogen-induced correlation in the trait data prior to performing GWAS on the host genomes. Broad-sense pathogen heritability H2 quantifies the fraction of total variance in a trait that is “inherited” from infection partner to infection partner, i.e., due to pathogen factors.''}
\end{enumerate}

\section*{Reviewer 2}

In this paper, Nadeau and colleagues look to correct for pathogen heritability in studies of host association with infectious diseases. They do this by extending an existing method which models trait evolution as a walk along a phylogeny, and provide a new method which which decomposes a pathogen trait into pathogen heritability vs 'other effects' using maximum-likelihood. By using only the component corresponding to 'other effects' as the phenotype in a standard host GWAS, they are able to correct for dependence between pathogen samples due to shared ancestry. As the authors note, this method is rather convenient as it results in an adjusted phenotype which can be used with standard GWAS tools. They apply their method to simulated data, showing a range (high heritability and selection) where a power advantage should be gained. They also apply the method to HIV set point viral load (spVL).

The paper is based on a sound idea, and is a technically competent solution, with no major scientific issues I could identify. Code is provided; data available via managed access; derivations in methods were appropriately explained. The introduction and conclusion are both well-written and do not overstate the results.

The main comment I have is based perhaps more on the impact of the study. The study is very strongly focused on analysis of HIV-1 spVL, which is not really the ideal trait for this method (as seen from the simulations in fig 2; and fig S4 showing the trait changing not all that much). It has long been established that there are two clear host associations, which indeed are exactly what the authors find before and after their adjustment. [If the aim was to try and find more associations for this trait, it would likely be much better to collect lots more samples than change the association methods, as has been found in host GWAS studies of other traits]. To increase the relevance of this study, the authors may consider whether they can apply their approach to other pathogens/traits. At the same time, I also appreciate that host/pathogen data is hard to come by and may have access issues that are too difficult to overcome. However, I do think if any of these studies in other pathogens could also be analysed here, it would really add to the paper. The authors cite a study of HCV as well as some other viral studies as examples, and I would also point to two more in bacteria (https://www.pnas.org/content/115/24/E5440; https://www.nature.com/articles/s41467-019-09976-3), and one in malaria (https://www.biorxiv.org/content/10.1101/2021.03.30.437659v1). Perhaps similar data is now available for SARS-CoV-2 also?

\textcolor{blue}{We were able to access additional data from the \emph{A. thaliana}-\emph{X. arboricola} pathosystem to demonstrate the broader applicability of our method. Unfortunately, neither dataset we analyzed had sufficiently highly correlated trait values amongst related pathogen strains to generate novel biological insights. We edited the text to emphasize that an advantage of our approach is that one gets information on the evolutionary dynamics of the trait in  the pathogen population, even if GWAS results are not changed.}

\vspace{5mm}
I also have some more minor comments:

\begin{enumerate}
    \item How does the method compare to the ATOMM approach in https://www.pnas.org/content/115/24/E5440? Can this be applied to any of the simulated or real data here for comparison. \\
    \textcolor{blue}{We could not apply ATOMM directly to our simulated data because we do not simulate pathogen genotypes, rather we simulate only pathogen trait values. ATOMM is also only implemented for haploid alleles, making it not ideal for the HIV data analyzed. Instead, we were able to access the same dataset ATOMM was tested on in their original publication. This allows us to compare GWAS results from our method versus the ATOMM method, which accounts for the pathogen and host genome jointly in the same linear mixed model and allows tests for marginal effects of host and pathogen alleles as well as pairwise host allele $\times$ pathogen allele interaction effects.}
    
    \textcolor{blue}{We present the findings of this comparison in the discussion: ``For comparison, we also fit the POUMM to quantitative disease resistance measurements from A. thaliana infected with the phytopathogenic bacteria X. arboricola. We estimated X. arboricola virulence heritability to be 33\% (95\% HPD 0 - 77\%). (Wang et al., 2018) originally estimated a QDR heritability of 44\% in this dataset, falling within the wide range of our estimate. We note that Wang et al. (2018) used a linear mixed model in which the experimental unit is QDR scored on individual leaves, whereas our estimate is based on much coarser binning of QDR scores into a mean score across all leaves on all host accessions and all replicates (N = 22)... For A. thaliana QDR to X. arboricola, the trait value correction does not utilize phylogenetic information because phylogenetic correlations between samples are too weak (maximum expected correlation between strains was $3.2 \times 10^{-12}$). We anyways corrected QDR trait values based on average QDR for each pathogen strain across the full range of host types. Results show slight decrease in p-values for the most-associated variants in this application as well, but the overall picture is consistent with previous GWAS results from Wang et al. (2018).''}
    
    \textcolor{blue}{We also highlight the linear mixed model approach presented in ATOMM as an alternate means to take pathogen effects into account: ``As in the A. thaliana-X. arboricola application, fitting the POUMM may reveal that expected phylogenetic correlations between samples are not strong enough to justify using our method to correct trait values in a GWAS. In this case, one may wish to use a linear mixed model as in (Wang et al., 2018), where the pathogen effect is co-estimated as a random effect.''}
    \item As I understand it the POUMM method assumes a 'correct' phylogeny. Do the authors have any thoughts about including uncertainty in the phylogeny (for example by using a posterior sample from e.g. mrbayes) and whether this may affect their approach? Relatedly, does a tree from just pol sequences (L342) give enough resolution? \\
    \textcolor{blue}{This is correct. We added a section ``Phylogenetic uncertainty'' in the Materials and Methods to describe an extra sensitivity test for the HIV analysis: ``Our method assumes the phylogeny accurately reflects the evolutionary relationships between pathogen strains... For the HIV application, we additionally tested the sensitivity of the inference to phylogenetic uncertainty. We inferred the phylogeny again, this time using the IQ-TREE option -wt to output all locally optimal trees. We fit the POUMM to two randomly selected trees from this set and repeated the trait correction and association testing steps using these trees and the corresponding POUMM parameter estimates.'' The results are briefly described in the Results section: ``We repeated the analysis using three different approximate maximum-likelihood phylogenies and these results were consistent (see Materials and Methods; Table S4).''}
    
    \item It would be useful to include more histograms and scatterplots of traits before/after the adjustment, as in figure S4. In particular, doing this plot again for one of the simulations where it does improve power would be informative. In my opinion, these could be a main figure. \\
    \textcolor{blue}{We implemented this suggestion and added two new main text figures (Figures 3 and 4). They are discussed in the section ``Estimator accuracy'': ``Figure 3 gives some intuition for how this correction works by contrasting simulated scenarios with high and low heritability and low selection strength/ low stochastic fluctuations. Depending on these parameters, trait values are more or less phylogenetically correlated (see also Figure 4) and the phylogeny is more or less useful for accurately estimating the heritable pathogen and corresponding non-heritable, non-pathogen part of the trait values.''}
    \item Do the authors know how the computational requirements of their ML estimation method scales with sample size? Would it work for a large SARS-CoV-2 phylogeny for instance? \\
    \textcolor{blue}{The POUMM R package for estimating the POUMM model parameters runs in less than an hour on a tree of 10,000 tips and has been shown to scale to trees of up to 100,000 tips. We additionally verified that the matrix inversion steps required for the last ML estimation step based on the fitted POUMM parameters ran successfully on a simulated tree of 10,000 tips using 4 cores with maximum memory use of 6500MB per core. We now highlight the method's scalability in the discussion: ``Our method relies on the freely available R package POUMM (Mitov and Stadler, 2017), which scales to trees of up to 10,000 tips (Mitov and Stadler, 2019).''}
    \item Fig S3 would be useful, but isn't really readable at present. Perhaps consider an interactive tree viewer to make this more useful. \\
    \textcolor{blue}{We added the same figure for the \emph{A. thaliana}-\emph{X. arboricola} dataset, which is much more legible due to having fewer samples (Figure S8). We also added sub-sampled phylogenies from the simulation study annotated with trait values in Figure 3 to demonstrate what phylogenetically-correlated trait values look like in a successful simulation example. Finally, we now reference the highest correlation values expected between any two tips in the HIV-1 and \emph{X. arboricola} phylogenies to add a quantitative element to the discussion of phylogenetic correlations: ``The highest expected correlation in trait values between any two tips in the HIV-1 phylogeny under the POUMM was 0.45. However, Figure S5 shows that this trait is not obviously phylogenetically structured in the cohort in general, despite high heritability.'' Then for the \emph{A. thaliana}-\emph{X. arboricola} application: ``Given the posterior mean estimates for the POUMM parameters, expected correlation in trait values between tips were very low (maximum value $3.2 \times 10^{-12}$ compared to maximum value of 0.45 in the HIV-1 spVL application). Thus, the phylogeny is not very informative for a trait value correction.''}
    \item The authors I think only look at power/TPR. What about the effect on false positive rate? \\
    \textcolor{blue}{We re-ran the simulations to assess the false positive rate and include the results in the Materials and Methods section: ``For our main results (Figure 2) we simulated 20 truly associated variants, as described above. To also check the false positive rate (FPR), we re-ran the simulations with an additional 80 non-associated variants. Across all the association tests in this second simulation setup (7 $H^2$ levels × 10 $\alpha$ levels × 100 variants × 20 replicates per scenario = 140,000 association tests), FPR was 0.0005 using the true (simulated) non-pathogen part of the trait, 0.0005 using the estimated non-pathogen part of the trait, and 0.0006 using the scaled total trait value. These rates are comparable to the expected FPR of 0.0005 at significance level 0.05 corrected for 100 tests. Given the stricter correction for multiple testing in this second simulation setup, the TPR decreased significantly across all three GWAS response variables used.''}
    \item I don't think panel A of figure 2 is explained well by the caption or text, I could not understand what the RMSE is showing here. What does 'more accurate' look like in this plot? \\
    \textcolor{blue}{We edited the figure legend to make it clearer: ``(A) shows the root mean squared error (RMSE) of our estimator (left) compared to un-corrected trait values, scaled by their mean (right) under each simulated evolutionary scenario. The RMSE is with reference to the true (simulated) host part of the trait values. Thus, more accurate estimates (lower RMSE) mean the trait value used for GWAS will be closer to the true host part of the trait value.''}
    \item Is there any explanation why the simulated (actual) and estimated non-pathogen trait in fig 2B start to look quite different at stronger selection strengths. \\
    \textcolor{blue}{We added a figure (Figure 4) to illustrate how the correlation between clustered tips on the phylogenetic tree decreases with increasing selection strength/ increasing stochasticity. Without phylogenetic signal, our method has no information to estimate the pathogen effect on the trait. This is described in the section ``Estimator accuracy'': ``Depending on these parameters, trait values are more or less phylogenetically correlated (see also Figure 4) and the phylogeny is more or less useful for accurately estimating the heritable pathogen and corresponding non-heritable, non-pathogen part of the trait values.''}
    \item Using plink and removing samples with divergent ancestry is probably a little outdated for host GWAS. A linear mixed model would likely provide more power still. In particular, I'd suggest warped-lmm (http://dx.doi.org/10.1038/ncomms5890), which also transforms the phenotype to increase power. Furthermore, I would be interested to see histograms/scatter plots of a few phenotypes: a) no adjustment; b) warped-lmm only; c) the author's method; d) both adjustments. Perhaps the methods are similar, or using both does increases power beyond either method alone. \\
    \textcolor{blue}{We highlight that our method is a pre-processing step before GWAS and can be combined with many different GWAS models implemented in softwares like PLINK, GEMMA, and warped-lmm. Our revised manuscript focuses more on the method rather than generating biological insights for HIV or \emph{X. arboricola}, so we feel that updating our example GWAS is beyond the scope of the current study. We added this explanation to the discussion: ``In cases where a correction can be estimated and applied using our method, the corrected trait values are envisioned to be used in any of the previously developed GWAS models for the actual association testing (we used a linear model approach implemented in PLINK (Chang et al., 2015), though a more advanced method would be to use a linear mixed model with host ancestry as a random effect).''}
\end{enumerate}

\vspace{5mm}
Suggested text changes:

\begin{enumerate}
    \item The first sentence of the abstract is probably also true for many common diseases analysed in human GWAS (apart from perhaps the pathogen, but gut microbiome may be a factor too). \\
    \textcolor{blue}{We changed the wording to make it clearer what sets infectious disease GWAS apart: ``Infectious diseases are particularly challenging for genome-wide association studies (GWAS) because genetic effects from two organisms (pathogen and host) can influence a trait.''}
    \item L84-86. There are more than two methods for this, see also https://doi.org/10.1371/journal.pgen.1007758; https://doi.org/10.1093/bioinformatics/bty539; https://doi.org/10.1099/mgen.0.000469. Maybe remove 'two', as this will change anyway.\\
    \textcolor{blue}{We removed the ``two'' as suggested: ``More recently, analogous methods have been developed for microbial GWAS, where clonal reproduction exacerbates population structure (Power et al., 2017). Microbial GWAS-specific phylogenetic methods to account for population structure in microbial GWAS include explicitly testing for lineage-specific effects as in Earle et al., (2016) and modified association tests that account for phylogenetic relationships amongst samples as in Collins and Didelot, (2018).''}
    \item L110 'Of these, only pathogen effects are heritable from one transmission partner to another'. This might be nit-picking but I could see this phrasing being misunderstood by some readers. Environment will be partially shared along a transmission chain, probably. Interaction effects could also be shared if e.g. hosts are likely to have the same ancestry.\\
    \textcolor{blue}{We improved the phrasing to be more accurate: ``GWAS typically stratify samples or include covariates to correct for host genetic factors or non-genetic factors that may be correlated with a trait value. This leaves pathogen genetic factors as a remaining source of correlation, since close transmission partners may be infected with very similar pathogen strains.''}
    \item Equation 1 onwards. Using 'e' for the environmental effects is easy to get confused with more standard use of the number e = 2.718, especially when raised to a power. Consider changing this symbol.\\
    \textcolor{blue}{We changed the variable to $\epsilon$ and also changed the variable for the pathogen part of the trait to $g$ to match previous work on the POUMM (e.g. \citep{Mitov2017a-POUMM})}.
    \item Please add the link to the github code in the framework section at the end of the methods.\\
    \textcolor{blue}{We added the link in the section ``Phylogenetic trait correction'': ``All the code used to implement the method is available at https://github.com/cevo-public/POUMM-GWAS.''}
    \item L283 'In simulations, we showed this is the case when heritability is high, selection strength is low, and trait values are not subject to strong stochastic fluctuations.' Could the authors give a biological example of this? Would antimicrobial resistance variation fit the bill?\\
    \textcolor{blue}{We have not fit the POUMM to data from enough systems to have a clear example of this, so instead we suggest other systems that might fall into this case: ``Future applications of our method might investigate other clinically significant disease traits and outcomes that are affected by both host and pathogen genetic factors, for instance Hepatitis B Virus-related hepatocellular carcinoma (An et al., 2018), Hepatitis C treatment success (Ansari et al., 2017), and susceptibility to or severity of certain bacterial infections, e.g. Donnenberg et al. (2015); Messina et al. (2016). Transcriptomic data has also previously been modeled as an evolving phenotype using an Ornstein-Uhlenbeck model (Rohlfs et al., 2014). Thus, one could also estimate pathogen effects on host gene expression.''}
    
    \textcolor{blue}{We anticipate antimicrobial resistance would be under strong selective pressure, but agree this could produce phylogenetically correlated trait values. We added a statement in the discussion: ``Finally, as we show here, our method is not anticipated to be useful in certain evolutionary scenarios. For instance, traits like antimicrobial resistance may be under strong selection pressure and be highly heritable. In these instances, our simulations do not point to a large improvement when adding our pre-processing step. In any case, such traits might violate the POUMM assumption that trait values vary as a random walk in continuous space if they are caused by few mutations of strong affect, meaning our approach would not apply. In this situation, one would rather account for antimicrobial resistance as a covariate in the GWAS association model.''}
\end{enumerate}
\end{document}